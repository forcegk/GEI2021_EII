\documentclass[a4paper,openright,12pt]{article}
\usepackage[utf8]{inputenc}
\usepackage{graphicx}
\usepackage{subfigure}
\usepackage[mathscr]{eucal}
\usepackage{titling}
\usepackage{float}
\usepackage{amsmath}
\usepackage{afterpage}
\usepackage{vmargin}
\usepackage[spanish,es-noshorthands]{babel}
\usepackage{csquotes}
\usepackage{eurosym} 
\usepackage{multirow}
\usepackage{xcolor}
\usepackage[export]{adjustbox}
\usepackage{url}
\usepackage[T1]{fontenc}
\usepackage{inconsolata}
\usepackage{listings}
\usepackage{minted}
\usepackage{array}
%\usepackage{amsfonts}
\usepackage{hyperref}
\usepackage{xurl}
\usepackage[backend=biber, style=authoryear, backref=true, hyperref=true, urldate=long]{biblatex}
\usepackage[none]{hyphenat}
\sloppy
\usepackage[document]{ragged2e}
\usepackage[shortlabels]{enumitem}
\usepackage{titlesec}
\setcounter{secnumdepth}{4}
\usepackage{datetime}
\usepackage{mfirstuc}
\setlist[enumerate]{itemsep=0mm}
\setlist[itemize]{itemsep=0mm}
\addbibresource{bibliography.bib}

% Packages for FSM and diagrams
\usepackage{pgf}
\usepackage{tikz}
\usetikzlibrary{arrows,automata}

% Things for Morse
\newcommand{\punto}{\kern+0.3pt\raisebox{0.35ex}{\huge\textbf.}}
\newcommand{\raya}{\kern+0.2pt\raisebox{-0.35ex}{\huge\textbf-}}

% Para generar imágenes mock
\usepackage{duckuments}

\setpapersize{A4}       %  DIN A4
\setmargins{3cm}        % margen izquierdo
{2cm}                   % margen superior
{15cm}                  % anchura del texto
{22.5cm}                % altura del texto
{10pt}                  % altura de los encabezados
{1cm}                   % espacio entre el texto y los encabezados
{0pt}                   % altura del pie de página
{2cm}                   % espacio entre el texto y el pie de página

\begin{document}

\author {Alonso Rodríguez}
\title {[Anteproyecto] Uso de ARM en el Datacenter}

% Título
\maketitle

% Justificamos el texto
\justifying{}


%%%%%%%%%%%%%%%%%%%%%%%%%%%%%%%%%%%%%%%%%%%%%%%%%%%%%%%%%%%%%%%%%%%%%%%%%%%%%%%%%%%%%%%%%%%%%%%%%%%%%%%%%%%%%%%%%%%%%%%%%
%                                                     DESCRIPCIÓN                                                       %
%%%%%%%%%%%%%%%%%%%%%%%%%%%%%%%%%%%%%%%%%%%%%%%%%%%%%%%%%%%%%%%%%%%%%%%%%%%%%%%%%%%%%%%%%%%%%%%%%%%%%%%%%%%%%%%%%%%%%%%%%
\section{Descripción}
Con el creciente interés en las arquitecturas RISC debido a su bajo consumo provocado por el auge de los dispositivos móviles,
los Datacenter, que también han ganado fuerza debido al creciente interés en los servicios cloud, también buscan una muy alta eficiencia
energética.

Esto ha llevado a que ARM, uno de los principales desarrolladores de CPUs RISC del mundo, cuyo modelo de negocio se basa en licenciar su propiedad intelectual
haya experimentado un crecimiento sin igual, y que no solo el mundo de los servidores sino también el de los consumidores (ref. Apple M1) estén virando hacia
esta plataforma.

No nos vamos a engañar, a mí personalmente lo que más me gusta es arquitectura, así que no es sorpresa que haya terminado eligiendo este tema (que espero sea adecuado
en el contexto de la asignatura).

En este trabajo tutelado se tratará el contexto histórico de las arquitecturas que se vienen usando en los CPD's y demás sistemas que satisfacen una alta capacidad de cómputo
o disponibilidad (algunos ya casi inexistentes, como mainframes, etc); se expondrán las ventajas y desventajas de cada arquitectura, costes de mantenimiento en función del
consumo/eficiencia (como por ejemplo coste de enfriamiento, etc).


%%%%%%%%%%%%%%%%%%%%%%%%%%%%%%%%%%%%%%%%%%%%%%%%%%%%%%%%%%%%%%%%%%%%%%%%%%%%%%%%%%%%%%%%%%%%%%%%%%%%%%%%%%%%%%%%%%%%%%%%%
%                                                   ÍNDICE PROPUESTO                                                    %
%%%%%%%%%%%%%%%%%%%%%%%%%%%%%%%%%%%%%%%%%%%%%%%%%%%%%%%%%%%%%%%%%%%%%%%%%%%%%%%%%%%%%%%%%%%%%%%%%%%%%%%%%%%%%%%%%%%%%%%%%
\clearpage
\section{Índice Propuesto}
Este índice es un índice provisional con títulos más significativos y detallados que tendrá la versión final, para poder intuir las líneas del trabajo tutelado.
\begin{enumerate}[1.]
\item Contexto histórico
\begin{enumerate}[1.]
    \item intel x86
    \item Otros: SPARC, Power, Itanium\ldots
    \item RISC vs CISC
\end{enumerate}

\item RISC y ARM
\begin{enumerate}[1.]
    \item RISC y el Consumo: Características
    \item Modelo de licenca: Ventajas
    \item Tendencia de mercado
\end{enumerate}

\item Problemas, obstáculos, realidades y promesas
\begin{enumerate}[1.]
    \item Problemas generales (disponibilidad de programas, etc)
    \item Consumidor <-- Adopción lenta y más ``dolorosa''
    \begin{enumerate}[1.]
        \item Software legacy
        \item Emulación y traducción
    \end{enumerate}
    \item Servidor <-- Adopción más rápida debido a entorno más cualificado
    \begin{enumerate}[1.]
        \item Menor consumo y temperatura
        \item Más ganancias \= Menores precios
        \item Port de programas más abundantes (se trata de otra manera al abandonware)
    \end{enumerate}
\end{enumerate}

\item AWS Graviton (Instancias ARM)
\begin{enumerate}[1.]
    \item Precios
    \item Ventajas e inconvenientes
\end{enumerate}

\item Clúster ARM con 8x Raspberry Pi 4B (Referencia al TFG) <-- Esto es una referencia a mi TFG, que creo que puede quedar bien en este ámbito. No es precisamente HA, pero algo es algo.

\item Futuro del desarrollo hardware y búsqueda de la eficiencia energética
\begin{enumerate}[1.]
    \item Reflexión final, (ref. Tendencia de mercado)
    \item Líneas futuras, expectativas cloud, crecimiento de datacenter, streaming\ldots
    \item Mención especial RISC-V (por ver si se comenta aquí o en un supuesto 3.4)
\end{enumerate}
    
\end{enumerate}

%%%%%%%%%%%%%%%%%%%%%%%%%%%%%%%%%%%%%%%%%%%%%%%%%%%%%%%%%%%%%%%%%%%%%%%%%%%%%%%%%%%%%%%%%%%%%%%%%%%%%%%%%%%%%%%%%%%%%%%%%
%                                                     BIBLIOGRAFÍA                                                      %
%%%%%%%%%%%%%%%%%%%%%%%%%%%%%%%%%%%%%%%%%%%%%%%%%%%%%%%%%%%%%%%%%%%%%%%%%%%%%%%%%%%%%%%%%%%%%%%%%%%%%%%%%%%%%%%%%%%%%%%%%
\clearpage
\section{Bibliografía}
Esto no es una bibliografía propiamente, ya que no he realizado ninguna cita y \LaTeX{} no me coloca nada aquí, pero por poner una lista de algunas páginas web consultadas:
\begin{itemize}
    \item \url{https://www.anandtech.com/show/15578/cloud-clash-amazon-graviton2-arm-against-intel-and-amd}
    \item \url{https://es.wikipedia.org/wiki/Arm_Holdings}
    \item \url{https://d1.awsstatic.com/events/reinvent/2019/REPEAT_1_Deep_dive_on_Arm-based_EC2_instances_powered_by_AWS_Graviton_CMP322-R1.pdf}
    \item Realmente no he consultado muchas URLs para la realización de este trabajo, porque en mi TFG ya llevo bastante investigado acerca de ARM.
\end{itemize}
%\clearpage
%\begin{flushleft}
%\printbibliography[]{}
%\end{flushleft}

\end{document}