\documentclass[a4paper,openright,12pt]{article}
\usepackage[utf8]{inputenc}
\usepackage{graphicx}
\usepackage{subfigure}
\usepackage[mathscr]{eucal}
\usepackage{titling}
\usepackage{float}
\usepackage{amsmath}
\usepackage{afterpage}
\usepackage{vmargin}
\usepackage[spanish,es-noshorthands]{babel}
\usepackage{csquotes}
\usepackage{eurosym} 
\usepackage{multirow}
\usepackage{xcolor}
\usepackage[export]{adjustbox}
\usepackage{url}
\usepackage[T1]{fontenc}
\usepackage{inconsolata}
\usepackage{listings}
\usepackage{minted}
\usepackage{array}
%\usepackage{amsfonts}
\usepackage{hyperref}
\usepackage{xurl}
\usepackage[backend=biber, style=authoryear, backref=true, hyperref=true, urldate=long]{biblatex}
\usepackage[none]{hyphenat}
\sloppy
\usepackage[document]{ragged2e}
\usepackage[shortlabels]{enumitem}
\usepackage{titlesec}
\setcounter{secnumdepth}{4}
\usepackage{datetime}
\usepackage{mfirstuc}
\setlist[enumerate]{itemsep=0mm}
\setlist[itemize]{itemsep=0mm}
\addbibresource{bibliography.bib}

% Packages for FSM and diagrams
\usepackage{pgf}
\usepackage{tikz}
\usetikzlibrary{arrows,automata}

% Things for Morse
\newcommand{\punto}{\kern+0.3pt\raisebox{0.35ex}{\huge\textbf.}}
\newcommand{\raya}{\kern+0.2pt\raisebox{-0.35ex}{\huge\textbf-}}

% Para generar imágenes mock
\usepackage{duckuments}

\setpapersize{A4}       %  DIN A4
\setmargins{3cm}        % margen izquierdo
{2cm}                   % margen superior
{15cm}                  % anchura del texto
{22.5cm}                % altura del texto
{10pt}                  % altura de los encabezados
{1cm}                   % espacio entre el texto y los encabezados
{0pt}                   % altura del pie de página
{2cm}                   % espacio entre el texto y el pie de página

\begin{document}

\author {Alonso Rodríguez}
\title {[Anteproyecto] Uso de ARM en el Datacenter}
\date {\today\\v2.0}

% Título
\maketitle

% Justificamos el texto
\justifying{}


%%%%%%%%%%%%%%%%%%%%%%%%%%%%%%%%%%%%%%%%%%%%%%%%%%%%%%%%%%%%%%%%%%%%%%%%%%%%%%%%%%%%%%%%%%%%%%%%%%%%%%%%%%%%%%%%%%%%%%%%%
%                                                     DESCRIPCIÓN                                                       %
%%%%%%%%%%%%%%%%%%%%%%%%%%%%%%%%%%%%%%%%%%%%%%%%%%%%%%%%%%%%%%%%%%%%%%%%%%%%%%%%%%%%%%%%%%%%%%%%%%%%%%%%%%%%%%%%%%%%%%%%%
\section{Descripción}\label{descr}
Con el creciente interés en las arquitecturas RISC debido a su bajo consumo provocado por el auge de los dispositivos móviles,
los Datacenter, que también han ganado fuerza debido al creciente interés en los servicios cloud, también buscan una muy alta eficiencia
energética.

Esto ha llevado a que ARM, uno de los principales desarrolladores de CPUs RISC del mundo, cuyo modelo de negocio se basa en licenciar su propiedad intelectual
haya experimentado un crecimiento sin igual, y que no solo el mundo de los servidores sino también el de los consumidores estén virando hacia esta plataforma.
\parencite{apple_m1_overview}

No nos vamos a engañar, a mí personalmente lo que más me gusta es arquitectura, así que no es sorpresa que haya terminado eligiendo este tema (que espero sea adecuado
en el contexto de la asignatura).

En este trabajo tutelado se tratará brevemente el contexto histórico de las arquitecturas que se vienen usando en los CPD's, el por qué de la entrada con fuerza de ARM en el
mismo ámbito, por razones de consumo/eficiencia (como por ejemplo coste de enfriamiento, etc), y se realizarán pruebas sintéticas sobre instancias ARM Graviton de AWS.\parencite{arm_aws_graviton_overview}

Finalmente se reflexionará (brevemente) acerca de la posible evolución del mundo de los CPD, cloud, streaming, etc.


%%%%%%%%%%%%%%%%%%%%%%%%%%%%%%%%%%%%%%%%%%%%%%%%%%%%%%%%%%%%%%%%%%%%%%%%%%%%%%%%%%%%%%%%%%%%%%%%%%%%%%%%%%%%%%%%%%%%%%%%%
%                                                   ÍNDICE PROPUESTO                                                    %
%%%%%%%%%%%%%%%%%%%%%%%%%%%%%%%%%%%%%%%%%%%%%%%%%%%%%%%%%%%%%%%%%%%%%%%%%%%%%%%%%%%%%%%%%%%%%%%%%%%%%%%%%%%%%%%%%%%%%%%%%
\clearpage
\section{Índice Propuesto}
Este índice es un índice provisional con títulos más significativos y detallados que tendrá la versión final, para poder intuir las líneas del trabajo tutelado.
\begin{enumerate}[1.]
\item Introducción
\begin{itemize}
    \item Breve introducción del contexto histórico, algo similar al apartado \ref{descr} de este anteproyecto.
\end{itemize}

\item Uso de ARM en el CPD
\begin{itemize}
    \item Workloads de servidor, consumo y costes asociados
    \item Modelo de licenca: Ventajas e inconvenientes
\end{itemize}

\item AWS Graviton (Instancias ARM)
\begin{itemize}
    \item Precios
    \item Ventajas e inconvenientes
    \item Pruebas reales sobre ARM Graviton
\end{itemize}

\item Futuro del desarrollo hardware y búsqueda de la eficiencia energética
\begin{itemize}
    \item Breve reflexión final, tendencias del mercado, expectativas cloud, crecimiento de datacenter, streaming\ldots
\end{itemize}
    
\end{enumerate}

%%%%%%%%%%%%%%%%%%%%%%%%%%%%%%%%%%%%%%%%%%%%%%%%%%%%%%%%%%%%%%%%%%%%%%%%%%%%%%%%%%%%%%%%%%%%%%%%%%%%%%%%%%%%%%%%%%%%%%%%%
%                                                     BIBLIOGRAFÍA                                                      %
%%%%%%%%%%%%%%%%%%%%%%%%%%%%%%%%%%%%%%%%%%%%%%%%%%%%%%%%%%%%%%%%%%%%%%%%%%%%%%%%%%%%%%%%%%%%%%%%%%%%%%%%%%%%%%%%%%%%%%%%%
\clearpage
\section{Bibliografía}
Esto no es una bibliografía propiamente, ya que estas son URLs de las que no he realizado ninguna cita y \LaTeX{} no me coloca nada aquí, pero por poner una lista
de algunas otras páginas web consultadas:
\begin{itemize}
    \item \url{https://www.anandtech.com/show/15578/cloud-clash-amazon-graviton2-arm-against-intel-and-amd}
    \item \url{https://es.wikipedia.org/wiki/Arm_Holdings}
    \item \url{https://d1.awsstatic.com/events/reinvent/2019/REPEAT_1_Deep_dive_on_Arm-based_EC2_instances_powered_by_AWS_Graviton_CMP322-R1.pdf}
    \item Realmente no he consultado muchas URLs para la realización de este trabajo, porque en mi TFG ya llevo bastante investigado acerca de ARM.
\end{itemize}
%\clearpage

\begin{flushleft}
\printbibliography[]{}
\end{flushleft}

\end{document}